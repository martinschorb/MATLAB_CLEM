\documentclass[10pt,a4paper,DIV17]{scrartcl}
% \areaset{210mm}{297mm}
\usepackage[T1]{fontenc}        %k.A.
\usepackage[latin1]{inputenc} %k.A.
%\usepackage{ngerman}   %Deutsch
\usepackage{amssymb,amsmath,amsthm}     %Mathe
\usepackage[pdftex]{graphicx} %Bilder
% \usepackage[german]{babel}
\usepackage[usenames,dvipsnames,pdftex]{color}
\usepackage[english]{babel}
\usepackage[font=small,labelfont=bf]{caption}
%\usepackage{bibgerm}
%\usepackage{subfig}
%\usepackage[subfigure]{tocloft}
\setlength{\headheight}{15pt}
\usepackage{fancyhdr}
\usepackage{listings}
\usepackage{svn-multi}
\svnidlong
{$HeadURL: https://svn.structures.embl.de/repo/schorb:Corr/doc/Correlate.tex $}
{$LastChangedDate: 2010-10-28 10:56:42 +0200 (Thu, 28 Oct 2010) $}
{$LastChangedRevision: 8 $}
{$LastChangedBy: schorb $}
\usepackage{url}



\usepackage{multicol}
\lstset{language=Matlab
, basicstyle=\ttfamily\scriptsize\bfseries
, keywordstyle=\color{Blue}\pmb
, breaklines=true
, commentstyle=\color{OliveGreen} 
, 
}
\lstset{numbers=left, stepnumber=2, numberstyle=\tiny, numbersep=5pt
%, frame=tb
}

\pagestyle{fancy}
\fancyhf{}
\lhead[\thepage]{\rightmark}
\rhead[\leftmark]{\thepage}

\title{Endocytosis Analysis - Algorithm overview}
\author{\small{corresponds to revision \svnrev\; commited at \svndate}\\\small{ of \url{martin_correlate.m} at \svnkw{HeadURL}}
}

\setlength{\columnsep}{10pt}

\begin{document}
 \maketitle

\section*{Vesicles}
\setcounter{section}{29} 
\begin{itemize}
 \item ellipsoid fit of vesicle coordinates using \url{ellipsoid_fit.m} (line 165)
 \item spherical fit of plasma membrane clicker using least quares fit (lines 198--201)
\end{itemize}
\section{Distance Vesicle Center - Membrane}
\begin{multicols}{2}
\lstinputlisting[firstline=225,
  lastline=235]{./martin_clickeranalysis1.m}

\columnbreak

\includegraphics*[width=.46\textwidth]{./sketches/ves_dist_mem.png}
\end{multicols}

\section{Vesicle surface area}
\vspace*{-4mm}\section{Vesicle volume}
\begin{multicols}{2}
\lstinputlisting[firstline=185,
  lastline=188]{./martin_clickeranalysis1.m}

  Thomsen approximation:\\ $      A\approx 4\pi\!\left(\frac{ (r_1 r_2)^{1.6}+(r_1 r_3)^{1.6}+(r_2 r_3)^{1.6} }{3}\right)^{0.625} $

   $ V = \frac{4}{3} \pi r_1 r_2 r_3$ 

\columnbreak

\includegraphics*[width=.46\textwidth]{./sketches/ves_surf_area.png}
\end{multicols}

\section{Teardrop presence?}\pagebreak
\section{Angle major axis - cell membrane}
\begin{multicols}{2}
\lstinputlisting[firstline=237,
  lastline=242]{./martin_clickeranalysis1.m}
\columnbreak
\includegraphics*[width=.46\textwidth]{./sketches/ves_ang_mjax_cm.png}
\end{multicols}

\section{Vesicle Axes, r\textsubscript{1} - \addtocounter{section}{1}\arabic{section}: r\textsubscript{2} - \addtocounter{section}{1}\arabic{section}: r\textsubscript{3} - \addtocounter{section}{1}\arabic{section}: Ellipticity $\frac{\mathbf{r_1}}{\mathbf{r_3}}$}
% \begin{lstlisting}
\begin{multicols}{4}\includegraphics*[width=.31\textwidth]{./sketches/ves_r1.png}\columnbreak
\includegraphics*[width=.31\textwidth]{./sketches/ves_r2.png}\columnbreak
\includegraphics*[width=.31\textwidth]{./sketches/ves_r3.png}
\end{multicols}

\section{Distance Vesicle center - transformed GFP position}
\begin{multicols}{2}
\lstinputlisting[firstline=289,
  lastline=294]{./martin_clickeranalysis1.m}
\vspace*{-24mm}
\lstinputlisting[firstline=305,
  lastline=306]{./martin_clickeranalysis1.m}
\columnbreak
\includegraphics*[width=.46\textwidth]{./sketches/ves_gfp_vesc.png}
\end{multicols}

\section{Angle Vesicle-GFP - major axis}
\begin{multicols}{2}
\lstinputlisting[firstline=307,
  lastline=308]{./martin_clickeranalysis1.m}
\columnbreak
\includegraphics*[width=.46\textwidth]{./sketches/ves_gfp_axis.png}
\end{multicols}

\section{Distance Vesicle center - transformed RFP position\\\addtocounter{section}{1}\arabic{section}\hspace{1mm} Angle Vesicle-RFP - major axis}\vspace*{-4mm}

\section{Angle teardrop - membrane axis}
\begin{multicols}{2}
\lstinputlisting[firstline=247,
  lastline=254]{./martin_clickeranalysis1.m}
\columnbreak
\includegraphics*[width=.46\textwidth]{./sketches/ves_trd_ang.png}
\end{multicols}
\section{- \addtocounter{section}{5}\arabic{section}\hspace{2mm} Likelihoods}
\begin{multicols}{2}
\lstinputlisting[linerange={259-287,310-319,375-382}]{./martin_clickeranalysis1.m}
\columnbreak
\includegraphics*[width=.46\textwidth]{./sketches/ves_likelihoods.png}\\\includegraphics*[width=.46\textwidth]{./sketches/ves_likelihoods2.png}
\end{multicols}


\newpage
\section*{Invaginations}
\setcounter{section}{49} 
\subsection*{Alignment(Code lines 426--618)}
\begin{itemize}
 \item Plasma membrane defined by first and last 3 points in clicker file
 \item Rotation 3D $\rightarrow$ flat plane (435--476)
 \item Rotation plasma membrane $\rightarrow x$-Axis, tip pointing up (478--476)
 \item Gaussian fit $\rightarrow$ height measure and centering of the profile
 \item Spline-fit to fill gaps
 \item local 2nd order polynomial fit
\end{itemize}

\subsection*{Key points}
\begin{center}\includegraphics[viewport=255 165 605 452,width=.50\textwidth]{./sketches/inv_sketch.pdf}
\end{center}

\subsubsection*{Neck}
To exclude the tip from the neck search, all points having a curvature $> -0.2$ are neglected.
Then the closest point of both left and right slopes is determined as "neck"
\lstinputlisting[firstline=660,
  lastline=670]{./martin_clickeranalysis1.m}

\subsubsection*{Cap}
Cap is defined by all points having a curvature $> -0.4$ and sit above half the invagination height.
\lstinputlisting[firstline=700,
  lastline=705]{./martin_clickeranalysis1.m}

\subsection*{Tip}
The point from within the cap subset located furthest apart from the center of the neck points is defined as tip.
\lstinputlisting[firstline=724,
  lastline=726]{./martin_clickeranalysis1.m}





\section{Invagination height}
\begin{multicols}{2}
\lstinputlisting[firstline=566,
  lastline=577]{./martin_clickeranalysis1.m}
\columnbreak
\includegraphics*[viewport=255 165 605 442,width=.40\textwidth]{./sketches/inv_height2.pdf}
\end{multicols}

\section{Neck presence}
\vspace*{-4mm}
\section{Neck height - \addtocounter{section}{1}\arabic{section}\hspace{1mm} Neck width}
\begin{multicols}{2}
\lstinputlisting[firstline=660,
  lastline=677]{./martin_clickeranalysis1.m}

To exclude the tip from the neck search, all points having a curvature $> -0.2$ are neglected.
Then the closest point of both left and right slopes is determined as "neck".
\columnbreak\\
\includegraphics*[viewport=232 169 582 446,width=.40\textwidth]{./sketches/inv_h_neck.pdf}
\newline\includegraphics*[viewport=232 169 582 446,width=.40\textwidth]{./sketches/inv_w_neck.pdf}
\end{multicols}
\pagebreak

\section{Distance neck to tip}
\begin{multicols}{2}
\lstinputlisting[firstline=700,
  lastline=727]{./martin_clickeranalysis1.m}
\columnbreak
\includegraphics*[viewport=232 169 582 446,width=.46\textwidth]{./sketches/inv_neck2tip.pdf}\\
Cap is defined by all points having a curvature $> -0.4$ and sit above half the invagination height.
\end{multicols}

\section{Curvature at neck}
\begin{multicols}{2}
\lstinputlisting[firstline=686,
  lastline=691]{./martin_clickeranalysis1.m}
\columnbreak
\includegraphics*[viewport=232 169 582 446,width=.46\textwidth]{./sketches/inv_neckcurv.pdf}
\end{multicols}

\section{Average height of points of maximum curvature}
\begin{multicols}{2}
\lstinputlisting[firstline=694,
  lastline=698]{./martin_clickeranalysis1.m}
\columnbreak
\includegraphics*[viewport=232 169 582 446,width=.46\textwidth]{./sketches/inv_maxcurv.png}
\end{multicols}

\pagebreak
\section{Axis angle}
\begin{multicols}{2}
\lstinputlisting[firstline=728,
  lastline=735]{./martin_clickeranalysis1.m}
\columnbreak
\includegraphics*[viewport=232 169 582 446,width=.46\textwidth]{./sketches/inv_axis_angle.pdf}
\end{multicols}

\section{Distance transformed (subtomo) GFP-coordinate - Neck / \addtocounter{section}{1}\arabic{section}\hspace{1mm} Tip / \addtocounter{section}{1}\arabic{section}\hspace{1mm} Membrane }
\begin{multicols}{2}
\lstinputlisting[firstline=970,
  lastline=973]{./martin_clickeranalysis1.m}
\includegraphics*[viewport=232 169 582 446,width=.46\textwidth]{./sketches/inv_dist_gfp_neck.pdf}

\columnbreak
\includegraphics*[viewport=232 169 582 446,width=.46\textwidth]{./sketches/inv_dist_gfp_tip.pdf}\\
\includegraphics*[viewport=232 169 582 446,width=.46\textwidth]{./sketches/inv_dist_gfp_mem.pdf}
\end{multicols}
\pagebreak
\section{Average curvature at the tip}
\begin{multicols}{2}
\lstinputlisting[firstline=726,
  lastline=733]{./martin_clickeranalysis1.m}
\columnbreak
Averages the curvature of the tip $\pm$ 10 datapoints to the respective right and left. (Not Gaussian curvature)
% \includegraphics*[viewport=232 169 582 446,width=.46\textwidth]{./sketches/inv_cap_curv.pdf}
\end{multicols}

\section{Surface area above neck positions}
\begin{multicols}{2}
\lstinputlisting[firstline=762,
  lastline=785]{./martin_clickeranalysis1.m}
\columnbreak
Calculates the surface area of the membrane patch residing above the neck position (here set to $y=0$)
% \includegraphics*[viewport=232 169 582 446,width=.46\textwidth]{./sketches/inv_cap_area.pdf}
\end{multicols}

\section{relative width at neck}
\begin{multicols}{2}
\lstinputlisting[firstline=787,
  lastline=796]{./martin_clickeranalysis1.m}
\columnbreak
Compares the neck width to the average width of the upper half of the invagination profile.
% \includegraphics*[viewport=232 169 582 446,width=.46\textwidth]{./sketches/inv_cap_area.pdf}
\end{multicols}

\section{Gaussian curvature at neck}
\begin{multicols}{2}
\lstinputlisting[firstline=918,
  lastline=932]{./martin_clickeranalysis1.m}
\columnbreak
\includegraphics*[viewport=232 169 582 446,width=.46\textwidth]{./sketches/inv_gaussneckcurv.pdf}
\end{multicols}
\pagebreak

\section{Height of vesicle surface area delimiters}
\begin{multicols}{2}
\lstinputlisting[firstline=867,
  lastline=892]{./martin_clickeranalysis1.m}
\columnbreak
% \includegraphics*[viewport=232 169 582 446,width=.46\textwidth]{./sketches/inv_cap_curv.pdf}
\end{multicols}

\section{Cap curvature}
\begin{multicols}{2}
\lstinputlisting[firstline=700,
  lastline=707]{./martin_clickeranalysis1.m}
\columnbreak
\includegraphics*[viewport=232 169 582 446,width=.46\textwidth]{./sketches/inv_cap_curv.pdf}
\end{multicols}

\section{Curvature Standard deviation at Cap - \addtocounter{section}{1}\arabic{section}\hspace{1mm} for entire profile}
\begin{multicols}{2}
\lstinputlisting[linerange={708-709}]{./martin_clickeranalysis1.m}
\columnbreak
% \includegraphics*[viewport=232 169 582 446,width=.46\textwidth]{./sketches/inv_cap_curv.pdf}
\end{multicols}

\section{Cap height - \addtocounter{section}{1}\arabic{section}\hspace{1mm} Cap width}
\begin{multicols}{2}
\lstinputlisting[firstline=717,
  lastline=718]{./martin_clickeranalysis1.m}
\columnbreak
% \includegraphics*[viewport=232 169 582 446,width=.46\textwidth]{./sketches/inv_cap_curv.pdf}
\end{multicols}

\section{GFP likelihood at neck, \addtocounter{section}{1}\arabic{section}\hspace{1mm} Tip, \addtocounter{section}{1}\arabic{section}\hspace{1mm} plasma embrane}
\begin{multicols}{2}
\lstinputlisting[firstline=970,
  lastline=976]{./martin_clickeranalysis1.m}
\columnbreak
% \includegraphics*[viewport=232 169 582 446,width=.46\textwidth]{./sketches/inv_cap_curv.pdf}
\end{multicols}


\end{document}